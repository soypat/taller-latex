\title{\LaTeX \,para Informes}
\author{Patricio Whittingslow}
\date{} %no me gusta que los profesores sepan que termine el informe ayer.
\documentclass{article}

\usepackage[spanish]{babel}
\usepackage[utf8]{inputenc}%Esto cambia la codificacion de caracteres. Sin esto no podes escribir texto con acentos normalmente.

%Cambio font a Helvetica

%\usepackage[scaled=1]{helvet}
%\usepackage[format=plain,
%            labelfont={bf,it},
%            textfont=it]{caption
%            }
%--------------------------------


%TITLE
\begin{document}

\maketitle
\begin{abstract}
Acá escribimos un resumen de lo que va tratar el informe. No tiene que ser muy largo pero si tiene que darle la idea al lector lo que se esta a punto de fumar.
\end{abstract}
\section{Introducción}
Bienvenidos a mi gran informe!
\subsection{Conceptos preliminares}
Lorem ipsum dolor sit amet, consectetur adipiscing elit, sed do eiusmod tempor incididunt ut labore et dolore magna aliqua. Ut enim ad minim veniam, quis nostrud exercitation ullamco laboris nisi ut aliquip ex ea commodo consequat. Duis aute irure dolor in reprehenderit in voluptate velit esse cillum dolore eu fugiat nulla pariatur. Excepteur sint occaecat cupidatat non proident, sunt in culpa qui officia deserunt mollit anim id est laborum.
\section{Conclusion}
Fin! La fisica es divertida \cite{serway}.

\begin{thebibliography}{9}
\bibitem{librosivorra}
https://www.uv.es/ivorra/Libros/Libros.htm

\bibitem{serway}
Física Moderna, \textit{Raymond A. Serway}

\end{thebibliography}


\begin{verbatim}
\begin{thebibliography}{9}
\bibitem{librosivorra}
https://www.uv.es/ivorra/Libros/Libros.htm

\bibitem{serway}
Física Moderna, \textit{Raymond A. Serway}

\end{thebibliography}
\end{verbatim}
\end{document}

